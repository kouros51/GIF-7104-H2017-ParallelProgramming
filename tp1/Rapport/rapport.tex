\documentclass{report}
\usepackage[french]{babel}
\usepackage{graphicx}
\usepackage[utf8]{inputenc}
\usepackage{float}
\usepackage{array}
\usepackage{tabularx}
\usepackage [T1]{fontenc}
\usepackage{xcolor}

\def\code#1{\texttt{#1}} % pour écrire du code (police monospace)
\def\comment#1{\color{gray} #1 \color{black}}

\begin{document}
\renewcommand{\labelitemi}{$\bullet$}
\thispagestyle{empty}

\begin{center}
	\vspace*{1cm}
	\huge  \bf Rapport du TP1: Paralèlliser l'algorithme du Crible d'Ératosthène\\
	\vspace{2.5cm}
	\normalsize
	\textit{Houssem Sebouai}\\
	111 134 915\\
	houssem.sebouai.1@ulaval.ca\\

  \vspace{1cm}
	\normalsize
	\textit{Thierry St-Gelais}\\
111 169 338\\
thierry.st-gelais.1@ulaval.ca\\

	\vspace{2cm}
	Dans le cadre des travaux pratiques du cours\\
	\LARGE GLO-7014\\
	\large Programmation parallèle et distribuée\\
	Professeur: Marc Parizeau

	\vfill
	\includegraphics[width=5cm]{Images/logo.jpg}
	\\
	Hiver 2017
\end{center}

\newpage

\tableofcontents
\listoffigures
\listoftables
\newpage
\chapter{Introduction}

	Le but de ce travail était réaliser une algorithme {\it multithread} (multifilaire)
	ayant pour but de trouver les nombre premiers inférieurs à un seuil fixé.
	Nous devions nous inspirer de l'algorithme du Crible d'Ératosthène.

\newpage
\chapter{Description de la méthode}

	\section{Explication de l'algorithme}
	
		L'algorithme du Crible d'Ératosthène consiste à parcourir un tableau 
		en ordre croissant, tout en éliminant successivement les multiples des 
		nombres rencontrés du tableau, ou à les {\it flagger} (marquer) comme 
		étant des nombres composés et donc, non-premiers.
		
	\bigskip		
	\section{Parralélisation de l'algorithme}

		Dans le cas présent, l'algorithme expliqué ci-haut est parralélisé de 
		manière à ce que chaque fil d'exécution prenne en charge un nombre. 
		Par exemple, si un premier fil s'occupe de tous les multiples de 2, 
		alors le fil suivant marquera tous les multiples de 3, et cetera.
		
		\bigskip
		\noindent ex.:
		
		\noindent
		\code{
			\comment{
			1 étant un nombre premier par défaut, il est possible de \\l'ignorer 
			dans l'élaboration du code suivant
			} \\	
			\\
			lN: entier, \comment{un entier correspondant à la limite de recherche 
			pour \\les nombres premiers}\\
			lT: entier, \comment{un entier correspondant au nombre de fils 
			d'exécution \\demandés} \\
			\\
			gCand: entier, \comment{Entier qui permet de déterminer les prochains
			\\nombres à marquer} \\
			\\
			N[lN] = [0,0,0, $\ldots$ ,0]: char, \comment{Un tableau de {\it flags}} \\
			\\
			T[lT] = [null, null, $\ldots$ , null]: ptr, \comment{Tableau contenant 
			des \\pointeurs vers les différents fils d'exécution} \\
			\\
			Pour (i=2..lT) \\
			\hspace*{16pt} T[i] = créer\_Fil(exec\_Crible(gCand))
		}
		
		\bigskip
		Ici, les fonctions \code{creer\_Fil()} et \code{exec\_Crible()}
		font exactement ce qu'elles prétendent: la première crée un nouveau fil 
		d'exécution qui prendra en charge la fonction passée en paramètre.
		
		\smallskip
		La seconde, elle, marque dans le tableau les multiples du nombre qui 
		lui est passé en paramètre. 
		

\end{document}
