\documentclass{report}
\usepackage[french]{babel}
\usepackage{graphicx}
\usepackage[utf8]{inputenc}
\usepackage{float}
\usepackage{array}
\usepackage{tabularx}
\usepackage[T1]{fontenc}
\usepackage{xcolor}
\usepackage{multirow}


\def\code#1{\texttt{#1}} % pour écrire du code (police monospace)
\def\comment#1{\color{gray} #1 \color{black}}

\begin{document}
\renewcommand{\labelitemi}{$\bullet$}
\renewcommand{\labelitemii}{$\circ$}
\thispagestyle{empty}

\begin{center}
	\vspace*{1cm}
	\huge  \bf Rapport du TP2: Rapport du TP2: Paralèlliser un algorithme de filtrage numérique basé
	sur des opérations de convolution\\
	\vspace{1cm}

	\LARGE Equipe numéro 2\\
	\vspace{1.5cm}
	\normalsize
	\textit{Houssem Sebouai}\\
	111 134 915\\
	houssem.sebouai.1@ulaval.ca\\

  \vspace{1cm}
	\normalsize
	\textit{Thierry St-Gelais}\\
111 169 338\\
thierry.st-gelais.1@ulaval.ca\\

	\vspace{2cm}
	Dans le cadre des travaux pratiques du cours\\
	\LARGE GLO-7014\\
	\large Programmation parallèle et distribuée\\
	Professeur: Marc Parizeau

	\vfill
	\includegraphics[width=5cm]{Images/logo.jpg}
	\\
	Hiver 2017
\end{center}

\newpage

\tableofcontents
\listoffigures
\listoftables
\newpage
\chapter{Introduction}
Dans ce deuxième travail pratique, nous avons été amener à paralléliser un algorithme
de filtrage numérique basé sur des opérations de convolution. Nous avons utilisée OpenMP
pour effectuer cette tache.
\chapter{Description générale de l'algorithme}
sur les pixels d'une image en entrée pour produire une image en output.
Cette operation de convulution est applique selon un noyeau ( un noyau flou ou un noyau identite).
\chapter{Approche adoptée pour la parallélisation }
\chapter{Expérimentations et résultats}
\chapter{Conclusion}

\end{document}
